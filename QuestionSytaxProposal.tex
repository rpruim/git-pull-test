\documentclass{ximera}
%% handout
%% nohints
%% space
%% newpage
%% numbers

%\input{preamble.tex} %% we can turn off input when making a master document

\outcome{Illustrate Question-Answer Formats.}

\outcome{Have a nice basic example to work from.}

\title{First example}

\begin{document}
\begin{abstract}
In this activity we see some examples.
\end{abstract}
\maketitle

\section{Numerical Answers}

\begin{exercise}
Let's see if we can do a little bit of arithmetic.
What is $3+4$? 
\numberAnswer[tol=+1e-6]{7}

\begin{hint}
Use your fingers.
\end{hint}
\end{exercise}


\section{Expression answers}

\begin{exercise}
Let's do some calculus.

What is $\frac{d}{dx} \sin(x)$?
\begin{prompt}
$\frac{d}{dx} = \expressionAnswer[test=evaluate]{\cos x}
\end{prompt}
\end{exercise}


\section{Mutliple choice}

Note: multiple correct answer are allowed.

\begin{question}
What are derivatives good for?
\begin{multipleChoice}
\choice[label=money]{Making lots of money by clever investment strategies.}
\choice[correct, label=change]{Determining how fast a function is changing.}
\choice[correct, label=optimize]{Finding the minimum value of a function.}
\end{multipleChoice}
\begin{hint}
There are only two correct answers.  This is math class.
\end{hint}
\end{question}

\section{Free reponse}

\begin{exercise}
How are you liking the Ximera project?
\begin{textAnswer}  % freeResopnse? should this be freeAnswer?  textAnswer? textBox?
Enter your response here.
\end{textAnswer}
\end{exercise}


\section{Inline text (short)}

\begin{exercise}
My name is \textAnswer[free]{} and I'm \textAnswer[free]{} years old.  My 
current mathematics class is \textAnswer[ignorespaces]{Calc1|Calc2}.     % can we handle regular expression-like thigs in the answers?
\end{exercise}

\section{Matix answer}

What is the inverse of $M = \begin{matrix}1 & 2 \\ 3 & 4\end{matrix}$?

\begin[test=equivalent, name=M]{matrixAnswer}
[[1,-1],[-1,2]]
\end{matrixAnswer}


Enter a matrix with determinant 1 that has no zero entries.

\begin[langauge=javascript]{code}
function unitDeterminant() {
  A[1][1] * A[2][2] - A[1][2] * A[2][1] == 1 && 
    A[1][1] != 0 &&
    A[1][2] != 0 &&
    A[2][1] != 0 &&
    A[2][2] != 0
}
\end{code}

\begin[test=unitDeterminant, name=A]{matrixAnswer}
\end{matrix-answer}
\end{matrixAnswer}

Now I'm dreaming a bit of a future world where 
Enter a matrix with determinant 1 that has no zero entries.

\begin[langauge=javascript]{code}
unitDeterminant <- function() {
  determinant(A) == 1 && all(A != 0)
}

\begin[test=unitDeterminant,name=A,langauge=R]{matrixAnswer}
\end{matrix-answer}
\end{matrixAnswer}


\section{Code answer}

\begin[language=python]{code}
def testFoo():
  foo(3,5) == (8,15)
\end{code}

Modify this funciton so that it returns a tuple containing both the sum
and product of its inputs.
\begin[language=phython, test=testFoo]{codeAnswer}
def foo(a,b):
  return a+b
\end{codeAnswer}

\section{Multi-stage exercises}

\begin{exercise}

\begin{stage}
Demonstrate that you can add by answering this simple arithmetic problem.

$ 3 + 5 = $ \numberAnswer{5}   % use 

\begin{solution}
One way to do this is to count-on from 5: 6-7-8
\end{solution}
\end{stage}

\begin{stage}
Now that you now how to add, let's see if you can multiply.
\end{stage}
$ 3 * 5 = $ \numberAnswer{15}   % use 

% feed back given upon succesful complete can be put into a stage with no answers.
\begin{stage} 
Great.  You can add and multiply.
\end{stage}

\end{exercise}

\section{Multi-item answers (list of blah)}


\section{Post-answer stuff}
It would be nice to repond to student answers flexibly.  This is complicated by
the number of kinds of answers and the variety of ways that answers can be wrong.


On the to-do list:
\begin{enumeate}
\item
list-of-blah type answer format
\item
more thought about things to do after an answer is supplied.
\end{enumerate}

\end{document}